\documentclass[11pt,a5paper,fleqn,leqno]{book}
\usepackage[utf8x]{inputenc}
\usepackage{ucs}
\usepackage{amsmath}
\usepackage{amsfonts}
\usepackage{amssymb}
\usepackage{graphicx}
\usepackage[top=2.5cm, bottom=2.5cm, left=1.25cm, right=1.25cm]{geometry}
\usepackage{ifpdf}
\ifpdf
  \DeclareGraphicsRule{*}{mps}{*}{}
\fi
\usepackage[danish]{babel}
\usepackage[pdftex,
	pdfauthor={Frank Bille},
	pdftitle={Formelsamling},
	bookmarks=true,
	colorlinks,
	linkcolor=black
]
{hyperref}
\author{Frank Bille}
\title{Formelsamling}

\begin{document}

\setlength{\parindent}{0cm}

\frontmatter

\thispagestyle{empty}
\setlength{\arrayrulewidth}{0.1cm}
\null
\vspace{3cm}
\begin{flushright}
  \begin{tabular}{r|}
    \rule{0pt}{2ex} \\
    \Huge \textsf{\textbf{Formelsamling}}
    \rule{0pt}{4ex} \\
    \large \textsf{Frank Bille Jensen}
    \rule{0pt}{3ex} \\
    \rule{0pt}{3ex} \\
    \rule{0pt}{3ex} \\
    \rule{0pt}{3ex} \\
    \rule{0pt}{3ex} \\
    \rule{0pt}{3ex} \\
    \rule{0pt}{3ex} \\
    \rule{0pt}{3ex} \\
  \end{tabular}
\end{flushright}
\cleardoublepage
\setlength{\arrayrulewidth}{0.4pt}

\tableofcontents

\mainmatter

\pagestyle{headings}

\chapter{Formler og regneregler}

\newpage

\section{Areal og omkreds}

\subsection{Rektangel}

\includegraphics{graph.6}
\\
\\
Højde $h$ \\
Grundlinje $g$ \\
Areal $A$
\begin{equation}
A = h \cdot g
\end{equation}

\subsection{Parallelogram}

\includegraphics{graph.7}
\\
\\
Højde $h$ \\
Grundlinje $g$ \\
Areal $A$
\begin{equation}
A = h \cdot g
\end{equation}

\subsection{Kvadrat}

\includegraphics{graph.5}
\\
\\
Sidelængde $a$ \\
Areal $A$
\begin{equation}
A = a \cdot a = a^2
\end{equation}

\subsection{Trekant}

\includegraphics{graph.4}
\\
\\
Højde $h$ \\
Grundlinje $g$ \\
Areal $A$ \\
Omkreds $O$
\begin{equation}
A = \frac{1}{2} \cdot h \cdot g
\end{equation}
\begin{equation}
O = |AB| + |BC| + |CA|
\end{equation}

\subsection{Trapez}

\includegraphics{graph.8}
\\
\\
Højde $h$ \\
Parallelle sider $a$ og $b$ \\
Areal $A$
\begin{equation}
A = \frac{1}{2} \cdot h \cdot (a + b)
\end{equation}

\subsection{Cirkel}

\includegraphics{graph.9}
\\
\\
Radius $r$ \\
Areal $A$ \\
Omkreds $O$
\begin{equation}
A = \pi \cdot r^2
\end{equation}

\begin{equation}
O = 2 \cdot \pi \cdot r
\end{equation}

\section{Brøker}

\begin{equation} \label{eq:taeller_naevner_det_samme}
\frac{a}{a} = 1
\end{equation}

\begin{equation} \label{eq:konstant_gange_med_broek}
a \cdot \frac{b}{c} = \frac{a \cdot b}{c} = \frac{a}{c} \cdot b
\end{equation}

\begin{equation} \label{eq:broeker_med_ens_naevnere_plusminus}
\frac{a}{b} \pm \frac{c}{b} = \frac{a \pm c}{b}
\end{equation}

\begin{equation} \label{eq:broeker_med_forskellige_naevnere_plusminus}
\frac{a}{b} \pm \frac{c}{d} = \frac{a \cdot d \pm c \cdot b}{b \cdot d}
\end{equation}

\begin{equation} \label{eq:broeker_ganget_sammen}
\frac{a}{b} \cdot \frac{c}{d} = \frac{a \cdot c}{b \cdot d}
\end{equation}

\begin{equation} \label{eq:forlaenge_en_broek}
\frac{a}{b} = 1 \cdot \frac{a}{b} = \frac{c}{c} \cdot \frac{a}{b} = \frac{c \cdot a}{c \cdot b}
\end{equation}

\begin{equation} \label{eq:broeker_divideret}
\frac{\frac{a}{b}}{\frac{c}{d}} = \frac{a}{b} \cdot \frac{d}{c}
\end{equation}

\newpage

\section{Procentregning}

Kapitalfremskrivning \\
Startkapital $K_0$ \\
Rentefod $r$ \\
Kapital $K$ efter $n$ terminer
\begin{equation}
K = K_{0} \cdot \left(1+r\right)^{n}
\end{equation}
\\
\\
Gennemsnitlig procentvis ændring $r$
\begin{equation}
1+r = \sqrt[n]{\left(1+r_{1}\right) \cdot \left(1+r_{2}\right) \cdot ... \cdot \left(1+r_{n}\right)}
\end{equation} 

\newpage

\section{Potensregneregler}

\begin{equation}
a^r = 1 \cdot a_0 \cdot a_1 \cdot ... \cdot a_r
\end{equation}

\begin{equation}
a^r \cdot a^s = a^{r+s}
\end{equation}

\begin{equation}
\dfrac{a^r}{a^s} = a^{r-s}
\end{equation}

\begin{equation}
\left(a^r\right)^s = a^{r \cdot s}
\end{equation}

\begin{equation}
\left(a \cdot b\right)^{r} = a^{r} \cdot b^{r}
\end{equation}

\begin{equation}
\left(\dfrac{a}{b}\right)^{r} = \dfrac{a^{r}}{b^{r}}
\end{equation}

\begin{equation}
a^{0} = 1
\end{equation}

\begin{equation}
a^{-r} = \dfrac{1}{a^{r}}
\end{equation}

\begin{equation}
\sqrt[r]{a} = a^{\frac{1}{r}}
\end{equation}

\begin{equation}
\sqrt[s]{a^{r}} = a^{\frac{r}{s}}
\end{equation}

\vfill

\section{Proportionalitet}

Proportionale størrelse $x$ og $y$
\begin{eqnarray}
y & =&  k \cdot x \\
\frac{y}{x} & = & k \nonumber
\end{eqnarray}
\\
\\
Omvendt proportionale størrelser $x$ og $y$
\begin{eqnarray}
y & = & c \cdot \frac{1}{x} \\
x \cdot y & = & c \nonumber
\end{eqnarray}

\section{Cirkel}

\includegraphics{graph.12}
\\
\\
Ligning for cirkel med centrum i $C(a,b)$ og radius $r$
\begin{equation}
(x-a)^2 + (y-b)^2 = r^2
\end{equation}

\newpage

\section{Koordinatsystem i planen}

\includegraphics{graph.1}

Afstanden $|AB|$ mellem to punkter $A(x_1,y_1)$ og $B(x_2,y_2)$
\begin{equation}
|AB| = \sqrt{(x_2-x_1)^2 + (y_2-y_1)^2}
\end{equation}
\\
\\
\includegraphics{graph.2}

Midtpunkt $M$ af linjestykke $AB$
\begin{equation}
M\left(\frac{x_1+x_2}{2},\frac{y_1+y_2}{2}\right)
\end{equation}

\vfill

\section{Linje}

\includegraphics{graph.3}
\\
\\
Ligninger for linjen $l$
\begin{eqnarray}
y & = & ax+b \\
y-y_1 & = & a(x-x_1) \nonumber
\end{eqnarray}
\\
\\
Hældningskoefficent (stigningstal) $a$ for linjen $l$
\begin{equation}
a = \frac{y_2 - y_1}{x_2 - x_1}
\end{equation}

\begin{equation}
a = \tan v
\end{equation}
\\
\\
\includegraphics{graph.10}
\\
\\
Ortogonale linjer $l$ og $m$
\begin{equation}
l \perp m \Leftrightarrow a \cdot c = -1
\end{equation}
\\
\\
\includegraphics{graph.11}
\\
\\
Afstand $\text{dist}(P,l)$ fra punktet $P(x_1,y_1)$ til linjen $l$ med ligningen $y = ax+b$
\begin{equation}
\text{dist}(P,l) = \frac{|ax_1 + b - y_1|}{\sqrt{a^2 + 1}}
\end{equation}

\vfill

\section{Parabel}

\includegraphics{graph.13}
\\
\\
Ligning for parabel med symmetriakse parallel med andenaksen
\begin{equation}
y = ax^2 + bx +c
\end{equation}
\\
\\
Toppunkt $T$
\begin{equation}
T\left(\frac{-b}{2a}, \frac{-d}{4a}\right), \quad \text{hvor} \quad d = b^2 - 4ac
\end{equation}
\\
Skæringspunkter $S_1$ og $S_2$ med førsteaksen
\begin{equation}
S_1\left(\frac{-b - \sqrt{d}}{2a}, 0\right), \quad S_2\left(\frac{-b + \sqrt{d}}{2a}, 0\right)
\end{equation}

\newpage

\section{Trekant}

\subsection{Ensvinklede trekanter}

\includegraphics{graph.14}

\begin{equation}
\frac{a_1}{a} = \frac{b_1}{b} = \frac{c_1}{c} = k
\end{equation}

\begin{eqnarray}
a_1 & = & k \cdot a \\
b_1 & = & k \cdot b \nonumber \\
c_1 & = & k \cdot c \nonumber
\end{eqnarray}

\vfill

\subsection{Retvinklet trekant}

\includegraphics{graph.15}
\\
\\
Pythagoras' læresætning 
\begin{equation}
a^2 + b^2 = c^2
\end{equation}

\begin{equation}
\sin A = \frac{a}{c}
\end{equation}

\begin{equation}
\cos A = \frac{b}{c}
\end{equation}

\begin{equation}
\tan A = \frac{a}{b}
\end{equation}

\vfill

\subsection{Vilkårlig trekant}

\includegraphics{graph.16}
\\
\\
Cosinusrelation
\begin{equation}
c^2 = a^2 + b^2 - 2ab\cos C
\end{equation}
\\
\\
Sinusrelation
\begin{equation}
\frac{a}{\sin A} = \frac{b}{\sin B} = \frac{c}{\sin C}
\end{equation}
\\
\\
Trekantens areal $T$ \\
Højden på $c$ = $h_c$
\begin{eqnarray}
T & = & \frac{1}{2}h_c a \\
& = & \frac{1}{2}ab\sin C \nonumber
\end{eqnarray}

\newpage

\section{Polynomier}

\subsection{Førstegradspolynomium}

\includegraphics{graph.17}
\\
\\
Førstegradspolynomium, lineær funktion $f$
\begin{equation}
f(x) = ax+b
\end{equation}

\newpage

\subsection{Andengradspolynomium}

\includegraphics{graph.18}
\\
\\
Andengradspolynomium $p$ med rødder $x_1$ og $x_2$
\begin{eqnarray}
p(x) & = & ax^2 + bx + c \\
 & = & a(x-x_1)(x-x_2) \nonumber
\end{eqnarray}
\\
\\
Rødder i andengradspolynomium $p$
\begin{equation}
x_1 = \frac{-b - \sqrt{d}}{2a}, \quad \frac{-b + \sqrt{d}}{2a}, \quad \text{hvor} \quad d = b^2 - 4ac
\end{equation}
\\
\\
Polynomium $p$ af grad $n$
\begin{equation}
p(x) = a_nx^n + a_{n-1}x^{n-1} + \dots + a_1x + a_0
\end{equation}

\vfill

\section{Logaritmefunktioner}

Grafer for $y = \log x$ og $y = 10^x$

\begin{equation}
y = \log x \Leftrightarrow x = 10^y
\end{equation}

Logaritmefunktionen $\log$ med grundtal 10

\begin{equation}
\log 10 = 1
\end{equation}

\begin{equation}
\log(a \cdot b) = \log a + \log b
\end{equation}

\begin{equation}
\log\left(\frac{a}{b}\right) = \log a - \log b
\end{equation}

\begin{equation}
\log(a^r) = r \cdot \log a
\end{equation}

Grafer for $y = \ln x$ og $y = e^x$

\begin{equation}
y = \ln x \Leftrightarrow x = e^y
\end{equation}

Den naturlige logaritmefunktion $\ln$ med grundtal $e$

\begin{equation}
\ln e = 1
\end{equation}

\begin{equation}
\ln (a \cdot b) = \ln a + \ln b
\end{equation}

\begin{equation}
\ln\left(\frac{a}{b}\right) = \ln a - \ln b
\end{equation}

\begin{equation}
\ln(a^r) = r \cdot \ln a
\end{equation}

\vfill

\section{Eksponentialfunktioner}

Eksponentialfunktion

\begin{equation}
f(x) = a^x
\end{equation}

Funktion, der er proportional med en eksponentialfunktion

\begin{equation}
f(x) = b \cdot a^x
\end{equation}

Bestemmelse af fremskivningsfaktor $a$

\begin{equation}
a = \sqrt[x_2-x_1]{\frac{y_2}{y_1}}
\end{equation}

En eksponentielt voksende funktion $f$ med fremskrivningsfaktor $a$, hvor $a > 1$, dvs. med vækstrate $r$, hvor $r > 0$

\begin{eqnarray}
f(x) & = & b \cdot a^x \\
 & = & b \cdot (1+r)^x \nonumber \\
 & = & b \cdot e^{kx}, \quad \text{hvor} \quad k = \ln a \nonumber
\end{eqnarray}

Fordoblingskonstant $T_2$

\begin{equation}
T_2 = x_2 - x_1
\end{equation}

\begin{equation}
T_2 = \frac{\log 2}{\log a} = \frac{\ln 2}{\ln a} = \frac{\ln 2}{k}
\end{equation}

\begin{equation}
b \cdot a^x = b \cdot e^{kx} = b \cdot 2^{\frac{x}{T_2}}
\end{equation}

En eksponentielt aftagende funktion $f$ med fremskrivningsfaktor $a$, hvor $0 < a < 1$, dvs. med vækstrate $-r$, hvor $r > 0$

\begin{eqnarray}
f(x) & = & b \cdot a^x \\
 & = & b \cdot (1-r)^x \nonumber \\
 & = & b \cdot e^{-kx}, \quad \text{hvor} \quad k = -\ln a \nonumber
\end{eqnarray}

Halveringskonstant $T_{\frac{1}{2}}$

\begin{equation}
T_{\frac{1}{2}} = x_2 - x_1
\end{equation}

\begin{equation}
T_{\frac{1}{2}} = \frac{\log\left(\frac{1}{2}\right)}{\log a} = \frac{\ln\left(\frac{1}{2}\right)}{\ln a} = \frac{\ln\left(\frac{1}{2}\right)}{-k}
\end{equation}

\begin{equation}
b \cdot a^x = b \cdot e^{-kx} = b \cdot \left(\frac{1}{2}\right)^{\frac{x}{T_{\frac{1}{2}}}}
\end{equation}

\section{Potentsfunktioner}

Potensfunktion

\begin{equation}
f(x) = x^a = e^{a\ln x}
\end{equation}

\begin{equation}
y = x^a \Leftrightarrow x = \sqrt[a]{y}
\end{equation}

Funktion, der er proportional med en potensfunktion

\begin{equation}
f(x) = b \cdot x^a
\end{equation}

Bestemmelse af eksponenten $a$

\begin{equation}
a = \frac{\log y_2 - \log y_1}{\log x_2 - \log x_1} = \frac{\ln y_2 - \ln y_1}{\ln x_2 - \ln x_1}
\end{equation}

Funktion $f$ defineres som $f(x) = b \cdot x^a$. $f(x)$ fremskrives med faktor $q^a$, når $x$ fremskrives med faktor $q$

\begin{equation}
f(q \cdot x) = q^a \cdot f(x)
\end{equation}

\vfill

\section{Trigonoetriske funktioner}

\begin{equation}
(\cos x)^2 + (\sin x)^2 = 1
\end{equation}

\begin{eqnarray}
\cos(x + 2\pi) & = & \cos x \\
\sin(x + 2\pi) & = & \sin x \nonumber
\end{eqnarray}

\begin{eqnarray}
\cos(-x) & = & \cos x \\
\sin(-x) & = & - \sin x \nonumber
\end{eqnarray}

\begin{eqnarray}
\cos(\pi - x) & = & - \cos x \\
\sin(\pi - x) & = & \sin x \nonumber
\end{eqnarray}

\begin{eqnarray}
\cos\left(\frac{\pi}{2}-x\right) & = & \sin x \\
\sin\left(\frac{\pi}{2}-x\right) & = & \cos x \nonumber
\end{eqnarray}

\begin{equation}
\tan x = \frac{\sin x}{\cos x}
\end{equation}

\begin{equation}
\tan(-x) = -\tan x
\end{equation}

\begin{equation}
\tan(x+\pi) = \tan x
\end{equation}

Specielle funktionsværdier

\begin{tabular}{c|c c c c c}
\hline
Grader       & $0^{\circ}$ & $30^{\circ}$ & $45^{\circ}$ & $60^{\circ}$ & $90^{\circ}$ \\
\hline
Radiantal    & $0$ & $\dfrac{\pi}{6}$ & $\dfrac{\pi}{4}$ & $\dfrac{\pi}{3}$ & $\dfrac{\pi}{2}$ \\
\hline
sin          & $0$ & $\dfrac{1}{2}$ & $\dfrac{\sqrt{2}}{2}$ & $\dfrac{\sqrt{3}}{2}$ & $1$ \\
\hline
cos          & $1$ & $\dfrac{\sqrt{3}}{2}$ & $\dfrac{\sqrt{2}}{2}$ & $\dfrac{1}{2}$ & $0$ \\
\hline
tan          & $0$ & $\dfrac{\sqrt{3}}{3}$ & $1$ & $\sqrt{3}$ & -- \\
\hline
\end{tabular}

\section{Differentialregning}

Ligning for tangent $t$ i punktet $A(x_0,f(x_0))$
\begin{equation}
y = f(x_0) + f'(x_0)(x-x_0)
\end{equation}
\\
\\
Approximerede førstegradspolynomium $p$ for $f$ i tallet $x_0$
\begin{equation}
p(x) = f(x_0) + f'(x_0)(x-x_0)
\end{equation}
\\
\\
Regneregler for differentiation
\begin{equation}
\left(f \pm g\right)'(x) = f'(x) \pm g'(x)
\end{equation}

\begin{equation}
\left(f \cdot g\right)'(x) = f'(x) \cdot g(x) + f(x) \cdot g'(x)
\end{equation}

\begin{equation}
\left(k \cdot f\right)'(x) = k \cdot f'(x)
\end{equation}

\begin{equation}
\left(\frac{f}{g}\right)'(x) = \frac{f'(x) \cdot g(x) - f(x) \cdot g'(x)}{\left(g(x)\right)^2}
\end{equation}

\begin{equation}
\left(f \circ g\right)'(x) = \left(f\left(g(x)\right)\right)' = f'\left(g(x)\right) \cdot g'(x)
\end{equation}

\section{Regneregler for integration}

I dette afsnit betegner $F$ en stamfunktion til $f$

\paragraph{Ubestemt integral}

\begin{equation}
\int f(x)\, \mathrm{d}x = F(x) + c
\end{equation}

\begin{equation}
\int \left(f(x) \pm g(x)\right)\, \mathrm{d}x = \int f(x)\, \mathrm{d}x \pm \int g(x)\, \mathrm{d}x
\end{equation}

\begin{equation}
\int k \cdot f(x)\, \mathrm{d}x = k \cdot \int f(x)\, \mathrm{d}x
\end{equation}

\begin{equation}
\int f(x) \cdot g(x)\, \mathrm{d}x = F(x) \cdot g(x) -  \int F(x) \cdot g'(x)\, \mathrm{d}x
\end{equation}

\begin{equation}
\int f(g(x)) \cdot g'(x)\, \mathrm{d}x = \int f(t)\, \mathrm{d}t \quad \mbox{hvor} \quad t = g(x)
\end{equation}

\paragraph{Bestemt integral}

\begin{equation}
\int_a^b f(x)\, \mathrm{d}x = \left[F(x)\right]_a^b = F(b) - F(a)
\end{equation}

\begin{equation}
\int_a^b f(x)\, \mathrm{d}x = \int_a^c f(x)\, \mathrm{d}x + \int_b^c f(x)\, \mathrm{d}x
\end{equation}

\begin{equation}
\int_a^b \left(f(x) \pm g(x)\right)\, \mathrm{d}x = \int_a^b f(x)\, \mathrm{d}x \pm \int_a^b g(x)\, \mathrm{d}x
\end{equation}

\begin{equation}
\int_a^b k \cdot f(x)\, \mathrm{d}x = k \cdot \int_a^b f(x)\, \mathrm{d}x
\end{equation}

\begin{equation}
\int_a^b k \cdot f(x)\, \mathrm{d}x = k \cdot \int_a^b f(x)\, \mathrm{d}x
\end{equation}

\begin{equation}
\int_a^b f(x) \cdot g(x)\, \mathrm{d}x = \left[F(x) \cdot g(x) \right]_a^b - \int_a^b F(x) \cdot g'(x)\, \mathrm{d}x
\end{equation}

\begin{eqnarray}
\int_a^b f(g(x)) \cdot g'(x)\, \mathrm{d}x & = & \int_{g(a)}^{g(b)} f(t)\, \mathrm{d}t  \quad \mbox{hvor} \quad t = g(x) \\
 & = & F(g(b)) - F(g(a)) \nonumber
\end{eqnarray}

\chapter{Eksempler}

\newpage

\section{Brøker}

Benytter: \eqref{eq:konstant_gange_med_broek}

\[4 \cdot \frac{5}{4} = \frac{4 \cdot 5}{4} = \frac{4}{4} \cdot 5 = 5\]
\\
\\
Benytter: \eqref{eq:broeker_med_forskellige_naevnere_plusminus}, \eqref{eq:broeker_ganget_sammen} og \eqref{eq:forlaenge_en_broek}

\[\frac{7}{2} - \frac{8}{6} = \frac{7 \cdot 6 - 8 \cdot 2}{2 \cdot 6} = \frac{42 - 16}{12} = \frac{26}{12} = \frac{13 \cdot 2}{6 \cdot 2} = \frac{13}{6} \cdot \frac{2}{2} = \frac{13}{6}\]
\\
\\
Benytter: \eqref{eq:broeker_ganget_sammen}, \eqref{eq:forlaenge_en_broek} og \eqref{eq:broeker_divideret}

\[\frac{\frac{3}{4}}{\frac{9}{4}} = \frac{3}{4} \cdot \frac{4}{9} = \frac{3 \cdot 4}{4 \cdot 9} = \frac{12}{36} = \frac{1 \cdot 12}{3 \cdot 12} = \frac{1}{3} \cdot \frac{12}{12} = \frac{1}{3}\]

\section{Potensregneregler}

\[2^7 = 1 \cdot 2 \cdot 2 \cdot 2 \cdot 2 \cdot 2 \cdot 2 \cdot 2 = 128\]

\[3^3 \cdot 3^4 = (3 \cdot 3 \cdot 3) \cdot (3 \cdot 3 \cdot 3 \cdot 3) = 3 \cdot 3 \cdot 3 \cdot 3 \cdot 3 \cdot 3 \cdot 3 = 3^7\]

\[2^0 = 5^0 = 3572954^0 = 1\]

\[\sqrt{3} \cdot 3^2 = 3^{\frac{1}{2}} \cdot 3^2 = 3^{\frac{1}{2}+2} = 3^{2\frac{1}{2}}\]

\[\sqrt{5}^2 = 5^{\frac{1}{2}^2} = 5^{\frac{1}{2} \cdot 2} = 5^1 = 5\]

\[\sqrt{5^2} = 5^{2^{\frac{1}{2}}} = 5^{2 \cdot \frac{1}{2}} = 5^1 = 5\]

\end{document}
